\documentclass[12pt, titlepage, a4paper]{article}
\usepackage{authblk}
\usepackage[margin=1in]{geometry}
\RequirePackage{amsthm,amsmath,amsfonts}
\RequirePackage{natbib}
\RequirePackage[colorlinks,citecolor=blue,urlcolor=blue]{hyperref}
\usepackage{booktabs,longtable}
\usepackage{graphicx}
\usepackage{xcolor}
\usepackage{epstopdf}
\usepackage{bm}
\usepackage[ruled, lined]{algorithm2e}
% \usepackage{algorithm}
% \usepackage{algpseudocode}
\usepackage{setspace}
\usepackage{lineno}
\linenumbers*[1]
%% patches to make lineno work better with amsmath
\newcommand*\patchAmsMathEnvironmentForLineno[1]{%
	\expandafter\let\csname old#1\expandafter\endcsname\csname 
	#1\endcsname
	\expandafter\let\csname oldend#1\expandafter\endcsname\csname 
	end#1\endcsname
	\renewenvironment{#1}%
	{\linenomath\csname old#1\endcsname}%
	{\csname oldend#1\endcsname\endlinenomath}}%
\newcommand*\patchBothAmsMathEnvironmentsForLineno[1]{%
	\patchAmsMathEnvironmentForLineno{#1}%
	\patchAmsMathEnvironmentForLineno{#1*}}%
\AtBeginDocument{%
	\patchBothAmsMathEnvironmentsForLineno{equation}%
	\patchBothAmsMathEnvironmentsForLineno{align}%
	\patchBothAmsMathEnvironmentsForLineno{flalign}%
	\patchBothAmsMathEnvironmentsForLineno{alignat}%
	\patchBothAmsMathEnvironmentsForLineno{gather}%
	\patchBothAmsMathEnvironmentsForLineno{multline}%
}

%notations
\newcommand{\Prob}{\bm{\mathrm{Pr}}}
\newcommand{\given}{\, \vert \,}

%\allowdisplaybreaks
%\linespread{1.01}

% User-specified new commands

%Comments
\newcommand{\pz}[1]{\textcolor{purple}{(PZ: #1)}}
\newcommand{\jy}[1]{\textcolor{orange}{JY: #1)}}

\title{Working title: TBD}

\author[1]{Sydney Louit}

\author[1]{Jun Yan}

\affil[1]{Department of Statistics, University of Connecticut, 
Storrs, CT 06269, USA}

\author[2]{Panpan Zhang}

\affil[2]{Department of Biostatistics, Vanderbilt University Medical 
	Center, Nashville, TN 37203, USA}

\begin{document}
	
\maketitle

\begin{abstract}
	TBA

\bigskip

\noindent{\bf Key words.} TBA

\end{abstract}

\doublespacing

\section{Introduction}
\label{sec:intro}
	
\pz{A general question that we shall keep in mind: Whatever method 
developed in this paper can be extended to directed networks?}

	
{\em Community detection} is a fundamental problem in network 
analysis with extensive applications in various research 
disciplines. \pz{Add application references here.} In general, there 
are two classes of methods for solving community detection	
problems. 
The first class is based on probabilistic models that are used to 
characterize network structures \pz{Add model-based method 
references}, whereas the second class aims to optimize some 
objective functions of graph-based criteria assessing the 
performance of each possible community structure \pz{Add 
metric-based method references}. 

	
Most of existing community detection methods focus only on the 
observed topological structure (i.e., network adjacency matrix), but 
overlook the node attributes or edge features that may be correlated 
with network structure. Moreover, it is of substantial interest to 
answer the question that whether accounting for the auxiliary 
information contained in nodes or edges helps improve community 
detection results. In response, there are several recent research 
papers targeting the integration of network topology and node 
attributes for solving community detection problems. 
\cite{yang2013community} developed a scalable algorithm called 
CESNA where binary-valued node attributes was considered. The 
algorithm (with modifications) was then extended to multilayer 
networks by \citet{contisciani2020community}. Independently, 
\citet{zhang2016community} proposed a joint community detection 
criterion for which node connections and node features were 
leveraged by a tuning parameter. Very recently, 
\citet{yan2021covariate} investigated sparse networks via a 
covariate regularized community detection algorithm, a variant 
of which was extended to multilayer networks by 
\citet{xu2022covariate}.
	
\section{Covariate-Dependent Heterogeneous Stochastic Block Model}
\label{sec:cdhsbm}

Let $G = (V, E)$ denote a network with node set $V$ and edge set 
$E$. The structure of $G$ is characterized by an associated 
adjacency matrix $\bm{A}:=(A_{ij})_{n \times n}$ with $n = |V|$ 
being the number of nodes. For each $i \in V$, let $\bm{Z}_i := 
\{Z_{i1}, Z_{i2}, \ldots, Z_{iK}\} \in \{0, 1\}^{K}$ be the 
membership vector for node $i$, where $Z_{ik} = 1$ indicates that 
node $i$ belongs to community $k \in \{1, 2, \ldots, K\}$ with $K$ 
unknown. It is conventional to model $\bm{Z}_i$ via a multinomial 
distribution given by
\[
\bm{Z}_i \sim \mathrm{Multinomial}(1, \bm{\alpha}_i),
\]
where $\bm{\alpha}_i := (\alpha_{i1}, \alpha_{i2}, \ldots, 
\alpha_{iK})$ is a hyperparameter vector subject to $\{\alpha_{ik} 
\in [0, 1]:\sum_{k = 1}^{K} \alpha_{ik} = 1\}$. For convenience, let 
$\bm{Z}$ be an $n \times K$ matrix collection all $\bm{Z}_i$'s. The 
standard {\em stochastic block 
model}~\citep[SBM,][]{nowicki2001estimation, abbe2018community} 
assumes that the connectivity of nodes depends on their membership 
information:
\[
\Prob(A_{ij}\given Z_i = k, Z_j = l) = \mathrm{Bernoulli}(B_{kl}),
\]
where $B_{kl} \in [0, 1]$ is the block-wise probability for 
communities $k, l \in \{1, 2, \ldots, K\}$. 


We extend the standard SBM from two aspects. First, we incorporate 
node level information into the model. Specifically, assume that, 
for each $i \in V$, there is a $q$-long observed covariate vector 
$\bm{X}_i$ containing some auxiliary information (about $i$) that 
may influence its connectivity pattern. For each pair of $i, j \in 
V$, define a dual measure 
\[S_{ij} = g(\bm{X}_i, \bm{X}_j) \in \mathbb{R}\] 
quantifying the node level covariate-dependent similarity between 
$i$ and $j$. Conditional on $Z_{i} = k$ and $Z_{j} = l$, $B_{kl}$ in 
the standard SBM is modeled by a logistic regression: 
\[
B_{kl} = \frac{\exp(\beta_0 + \beta_{kl} S_{ij})}{1 + \exp(\beta_0 + 
\beta_{kl} S_{ij})},
\]
where $\beta_0$ presents the commonality across all the communities 
and $\beta_{kl}$ indicates the homogeneity of edge connection 
preserved for each pair of communities. 


Second, we consider node level latent factors that present node 
connection tendency reflecting node heterogeneity. For each $i \in 
V$, let $\theta_i$ be the associated random effect presenting node 
heterogeneity. Then, given $Z_i = k$ and $Z_j = l$, $B_{kl}$ is 
updated to
\[
B_{kl} = \frac{\exp(\beta_0 + \beta_{kl} S_{ij} + \theta_i + 
\theta_j)}{1 + \exp(\beta_0 + \beta_{kl} S_{ij} + \theta_i + 
\theta_j)},
\]
where large $\theta_i$ and $\theta_j$ jointly promote the connection 
between nodes $i$ and $j$.

\begin{align*}
\Prob(\bm{A} \given \bm{Z}, \bm{\beta}, K; \bm{S}, \bm{\theta}) = 
\prod_{1 \le i < j \le n}\prod_{1 \le k, l \le K}\Prob(A_{ij}\given 
Z_i = k, Z_j = l),
\end{align*}
where $\bm{\beta}$ is a collection of regression coefficients 
$\beta_0$ and all $\beta_{kl}$'s.

\pz{For the rest,
\begin{enumerate}
\item To estimate $K$, you can use the approach 
in~\citet{geng2019probabilistic}
\item For the prior of $\bm{\alpha}$, you may consider Dirichlet 
(search some references)
\item For the prior of $\bm{\beta}$, you may consider MVN; 
see~\citet{handcock2007model}
\end{enumerate}
}

\pz{Quick extensions:
\begin{enumerate}
\item Directed network: The likelihood will look similar, just 
$A_{ij} \neq A_{ji}$. Therefore, the product will be on $\prod_{1 
\le i \neq 
j \in n}$.
\item Multilayer network: The likelihood will look similar, just 
adding a super script $m \in \{1, 2, \ldots, M\}$. There will be an 
outer product in likelihood: $\prod_{m = 1}^{M}$.
\end{enumerate}
}

\jy{To get started with the simulation, let's generate covariate vector $\bm X$
  from a multivariate normal distribution. }

\bibliographystyle{chicago}
\bibliography{node_feature}
	
\end{document}

